\section{Related Work}
\label{sec:related_work}
\para{Multi-turn jailbreaks and escalation.} A growing body of work shows that gradual, multi-turn escalation can raise attack success rates compared to single-turn prompts. Crescendo introduced a progressive reveal strategy \cite{russinovich2024crescendo}, followed by variants such as Foot-In-The-Door (FITD) \cite{weng2025fitd}, reasoning-augmented conversation (RACE) \cite{ying2025race}, and tree-search jailbreaks such as Tempest \cite{zhou2025tempest}. These attacks motivate turn-indexed safety evaluation but do not isolate whether increased success reflects general prior regression or specific escalation tactics.

\para{Human red-teaming and multi-turn datasets.} Human-in-the-loop efforts find that most successful jailbreaks require multiple turns, and release datasets that capture realistic, long-horizon interactions \cite{li2024mhj}. Benchmarks such as \jbb and \wildjb provide standardized harmful prompts for evaluation, while \mtbench provides multi-turn benign tasks. Our study leverages these datasets to create controlled turn-count comparisons.

\para{Drift analyses and mitigations.} Representation analyses suggest that multi-turn histories can shift model state toward benign regions even when harmful intent emerges late \cite{bullwinkel2025repe}. Other work reframes multi-turn failures as intent mismatch and proposes mediator architectures \cite{liu2026intent}. Multi-turn safety alignment methods such as MTSA and STREAM target robustness via additional training and safety reasoning \cite{guo2025mtsa,kuo2025stream}. Goal tracking and policy optimization approaches further target drift in multi-turn settings \cite{coscia2025ongoal,wang2026icpo}. Unlike these contributions, we focus on a simple, controlled measurement of turn-indexed compliance and refusal across safety and benign tasks.
